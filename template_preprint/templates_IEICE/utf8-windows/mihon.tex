\documentclass[a4j,10pt]{jarticle}

%\usepackage{graphics} % EPS graphics
\pagestyle{empty} % ページ番号なし

\setlength{\Cvs}{1mm} % 章題前後の空行幅基準値
\setlength{\textheight}{247mm} % 本文領域縦幅
\setlength{\textwidth}{174mm} % 本文領域横幅
\setlength{\oddsidemargin}{-6mm} % 左マージン調整(奇数ページ)
\setlength{\evensidemargin}{-6mm} % 左マージン調整(偶数ページ)
\setlength{\topmargin}{-12mm} % 上マージン調整
\setlength{\columnsep}{8mm} % 段間隔調整

\begin{document}

\twocolumn[%
\begin{center}
{\LARGE\bf ○○に関する研究} \\
{\large A Study on XXX} \\
\vspace{1mm}
{\large 第壱 著者$^1$、第弐 著者$^1$、第参 著者$^2$、第四 著者$^2$} \\
{First Author $^1$, Second Author $^1$,
 Third Author $^2$, Fourth Author $^2$} \\
\vspace{1mm}
{\large $^1$著者所属1、$^2$著者所属2} \\
{$^1$ First Affiliation, $^2$ Second Affiliation} \\
\vspace{4mm}
\end{center}
]

\section{はじめに}

「はじめに」の本文。「はじめに」の本文。「はじめに」の本文。
「はじめに」の本文。「はじめに」の本文。「はじめに」の本文。
「はじめに」の本文。「はじめに」の本文。「はじめに」の本文。
「はじめに」の本文。「はじめに」の本文。「はじめに」の本文。

「はじめに」の本文。「はじめに」の本文。「はじめに」の本文。
「はじめに」の本文。「はじめに」の本文。「はじめに」の本文。
「はじめに」の本文。「はじめに」の本文。「はじめに」の本文。
\cite{1stPaper}

\section{方法}

「方法」の本文。「方法」の本文。「方法」の本文。
「方法」の本文。「方法」の本文。「方法」の本文。
「方法」の本文。「方法」の本文。「方法」の本文。
「方法」の本文。「方法」の本文。「方法」の本文。
\begin{equation}
A+B=C
\end{equation}
「方法」の本文。「方法」の本文。「方法」の本文。
「方法」の本文。「方法」の本文。「方法」の本文。
「方法」の本文。「方法」の本文。「方法」の本文。
「方法」の本文。「方法」の本文。「方法」の本文。
\cite{1stPaper,2ndPaper}

\section{結果と考察}

「結果と考察」の本文。「結果と考察」の本文。「結果と考察」の本文。
「結果と考察」の本文。「結果と考察」の本文。「結果と考察」の本文。
「結果と考察」の本文。「結果と考察」の本文。「結果と考察」の本文。
「結果と考察」の本文。「結果と考察」の本文。「結果と考察」の本文。

\begin{table}[tbh]
\begin{center}
\renewcommand{\tablename}{Table}
\setlength{\abovecaptionskip}{0mm} % 表キャプション上空行幅調整
\caption{表キャプション}
\begin{tabular}{lcl}
\hline
A & 1.0 & data 1 \\
B & 2.0 & data 2 \\
\hline
\end{tabular}
\end{center}
\end{table}

「結果と考察」の本文。「結果と考察」の本文。「結果と考察」の本文。
「結果と考察」の本文。「結果と考察」の本文。「結果と考察」の本文。
「結果と考察」の本文。「結果と考察」の本文。「結果と考察」の本文。
「結果と考察」の本文。「結果と考察」の本文。「結果と考察」の本文。
\cite{1stPaper,2ndPaper,3rdPaper}

\begin{figure}
%\includegraphics{fig01.eps}
  \setlength{\unitlength}{1mm}
  \begin{center}
  \begin{picture}(80,30)(0,0)
  \put(0,0){\framebox(80,30)}
  \end{picture}
  \end{center}
\setlength{\abovecaptionskip}{0mm} % 図キャプション上空行幅調整
\setlength{\belowcaptionskip}{0mm} % 図キャプション下空行幅調整
\renewcommand{\figurename}{Fig.}
\caption{図キャプション}
\end{figure}

「結果と考察」の本文。「結果と考察」の本文。「結果と考察」の本文。
「結果と考察」の本文。「結果と考察」の本文。「結果と考察」の本文。
「結果と考察」の本文。「結果と考察」の本文。「結果と考察」の本文。
「結果と考察」の本文。「結果と考察」の本文。「結果と考察」の本文。

\section{まとめ}

「まとめ」の本文。「まとめ」の本文。「まとめ」の本文。
「まとめ」の本文。「まとめ」の本文。「まとめ」の本文。
「まとめ」の本文。「まとめ」の本文。「まとめ」の本文。
「まとめ」の本文。「まとめ」の本文。「まとめ」の本文。

「まとめ」の本文。「まとめ」の本文。「まとめ」の本文。
「まとめ」の本文。「まとめ」の本文。「まとめ」の本文。
「まとめ」の本文。「まとめ」の本文。「まとめ」の本文。
「まとめ」の本文。「まとめ」の本文。「まとめ」の本文。
\cite{1stPaper,2ndPaper,3rdPaper}

\section*{謝辞}

本研究は○○による補助を受けて行なわれた。

\begin{thebibliography}{0}
\setlength{\itemsep}{0mm}
\setlength{\baselineskip}{10pt}
\vspace{2mm}
\bibitem{1stPaper} F. Author:
 Proc. Joint Conf. Hokkaido Capters, {\bf 1}, xxx--xxx (2007)
\bibitem{2ndPaper} S. Author and F. Author:
 Proc. Joint Conf. Hokkaido Capters, {\bf 2}, xxx--xxx (2007)
\bibitem{3rdPaper} T. Author, S. Author and F. Author:
 Proc. Joint Conf. Hokkaido Capters, {\bf 3}, xxx--xxx (2007)
\end{thebibliography}

\end{document}
