\begin{jabstract}

    kudo研では、5年次卒業論文、専攻科学士論文、要旨の制作には、texが使われてきました。
    毎年12月に入ると、研究も大詰めになり
    
    「さぁ、そろそろ論文書かなきゃなぁ…。」

    「えーとたしかkudo研は、texで書かないといけないんだよな。」

    「いいーーゃ、texってなんじゃぼけぇコラァ!!」\\
    ってなって、素人には、わけのわからないtex環境構築が始まります。
    私が5年生の時はそうでした。
    
    「こんな12月は、もう終わりにしたい!!」\\
    と思い、専攻科1年12月に念願のkudolab\_\LaTeX というものGithub上に作成することができました。
    kudo研に入ってくれた皆さんが簡単に環境構築できるようにまとめたものです。
    うまく環境構築できなかったら私のせいです。本当にごめんなさい。
    ですが、私はみなさんが思う100倍バカです。
    そんな私でも、環境構築できました。
    なのであきらめずに頑張ってください。

    今回私が提供するのは、"VScodeのLaTeX Workshopと実行ファイルLaTeXmkrcを用いたp\LaTeX とpbibtexの環境構築"となります。
    自分のパソコンに環境構築するのがベストです。
    研究室のパソコンに環境構築してもいいですが、居残り地獄になります(経験済み)。
    もし、自分のパソコンが無いという方は、家でメモ帳などに文章を考えて学校でコンパイルするのがいいでしょう。

    このテンプレートには、最低限の論文をtexで書くための情報を書いてあります。
    このテンプレートフォルダ使えば、しっかりとした論文をかけるでしょう。

    このkudolab\_\LaTeX は決して、私1人で作り上げたのものではありません。
    土台となるスタイルファイルは先輩から引き継いだものです。
    研究は一人ではできません。
    多くの先人たちの苦労があったことを忘れないでください。
    そして、みなさんも ” 後輩につなぐ ” という意識を持って、研究に励んでください。\\

    宇野慎太郎(shintaro.great0201@gmail.com) UNOSTOP!
    
\end{jabstract}
