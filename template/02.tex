\chapter{数式・特殊文字}
\vspace{20mm}\hrulefill\\\vspace{20mm}

この章では、数式・特殊文字の書き方について、記載する。
前章でもお伝えしたが、ここに記載しているのはあくまでも最低限です。
詳しくTexについて学びたい方は、Googleで"謎肉 スースキス カップラーメン"などで検索して、♪勝手に スースキス スキスしてください。

\newpage

\section{数式}
LaTeXで数式を書くためには、いくつかの方法があります。
以下に、基本的な数式の書き方を示す。
\begin{enumerate}
	\item インライン数式
	
		インライン数式は、テキスト内に数式を埋め込む場合に使用する。
		数式を $ ... $で挟んで使います。
		
		ex.これはインライン数式 $E=mc^2$ です。

	\item ディスプレイ数式
	
		ディスプレイ数式は、独立した行に数式を表示する場合に使用されます。
		'equation'環境を使用します。
		
		ex.これはディスプレイ数式です:
		\begin{equation}
			\int_{0}^{\pi} \sin(x) \, dx = 2
		\end{equation}

		\begin{eqnarray}
			E_0 = {K}\cdot{I_0}\cdot{R_L}\ /\ {n}\  [\ \mathrm{V_{rms}}\ ]
		\end{eqnarray}

\end{enumerate}

\section{特殊文字}
Latexにはたくさんの特殊文字があります。
全部書ききれるわけがないので、適当にリストアップしたものを示します。
その他、コードを忘れた場合は、VScode左バーにある"TEX">"SNIPET VIEW"から選択するのもいいだろう。

\begin{enumerate}
	\item 数学記号と演算子
	
	$\alpha, \beta, \gamma, \theta, \pi, \sqrt{x}, \frac{1}{2}, \sum_{i=1}^{n} x_i, \int_{a}^{b} f(x) \, dx$

	\item 上付き文字と下付き文字

	$x^2, a_{ij}, e^{i\theta}, \lim_{x \to \infty}$

	\item 分数
	
	$\frac{a}{b}, \frac{\sqrt{2}}{2}$

	\item 行列
	
	$\begin{bmatrix}
		1 & 2 \\
		3 & 4
	\end{bmatrix}$

\end{enumerate}